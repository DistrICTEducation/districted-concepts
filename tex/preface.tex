\chapter*{Preface}\label{chapter:preface}

\section*{Goal}

The goal of this text is to provide some background on various topics surrounding information technology.

Information technology can be approached from a very formal, mathematical perspective, but at the same time, a great deal about information technology is highly pragmatic and is closely related to the business side of things. In this text we will mainly use an informal way to describe things. Nonetheless, we will try to reference academic resources and scientific papers if possible.

Using this text one should be able to get an idea of 


\section*{Organization}

The organization of this document is as follows:

\begin{enumerate}[I.]
	\item \textbf{\autoref{part:programming-fundamentals} Programming fundamentals} : In this part we introduce the fundamentals of computer programming such as experessions, data structures and algorithms. We also provide a more mathematical approach to reason about programs, introducing time complexity of algorithms and related complexity problems.
		\begin{enumerate}[1.]
			\item \textbf{\autoref{chapter:formal-languages} Formal languages} : Introducing time complexity, computability, and formal languages.
			\item \textbf{\autoref{chapter:datastructures} Data structures} : Introducing various representations of data and how these data structures affect performance.
			\item \textbf{\autoref{chapter:algorithms} Algorithms} : Taking off from data structures introduced in the previous chapter, we take a closer look at a range of algorithms and how they are applied in different contexts. We will also try to give an introduction to certain general problems and how they can be solved algorithmically.
		\end{enumerate}

%	\item \textbf{\autoref{part:programming-language-concepts} Programming language concepts} : In this part we take a closer look at the various choices for designing programming languages and how these affect the bahaviour of the code written in these languages.
%		\begin{enumerate}[]
%			\item \textbf{\autoref{chapter:data} Data} : Introduction to primitives and data types.
%			\item \textbf{\autoref{chapter:references} References} : Overview of different paradigms of how data can be stored and referenced.
%			\item \textbf{\autoref{chapter:continuations} Continuations} : Introduction to how the control state of a program can be managed.
%			\item \textbf{\autoref{chapter:encapsulation} Encapsulation} : Introduction to different ways in which the scope of variables can be handled.
%		\end{enumerate}


	\item \textbf{\autoref{part:systems-and-architectures} Systems and architectures} : In this part we describe the systems and architectures on which the software is run. First, we describe operating systems, next computer networks, and finally distributed systems operating on these networks.
		\begin{enumerate}[1.]
			\item \textbf{\autoref{chapter:operating-systems} Operating systems} : Introduction to the mechanics of operating systems.
			\item \textbf{\autoref{chapter:computer-networks} Computer networks} : Introduction to the different models and protocols used in computer networks.
			\item \textbf{\autoref{chapter:distributed-systems} Distributed systems} : Introduction to various concepts related to distributed systems.
		\end{enumerate}


	\item \textbf{\autoref{part:security} Security} : This part deals with security concepts of software and networks.
		\begin{enumerate}[1.]
			\item \textbf{\autoref{chapter:security-introduction} Introduction to security} : An introduction to some terminology regarding software security.
			\item \textbf{\autoref{chapter:low-level-security} Low-level security} : An overview of some attack and defense techniques for low-level programming languages.
			\item \textbf{\autoref{chapter:access-control} Access control} : An overview concepts used in access control and authentication.
			\item \textbf{\autoref{chapter:web-application-security} Web application security} : An overview of some attack and defense techniques for web applications.
		\end{enumerate}


	\item \textbf{\autoref{part:software-design} Software design} : This part deals with the design of software, software architecture and user interface design.
		\begin{enumerate}[1.]
			\item \textbf{\autoref{chapter:requirements-analysis} Requirement analysis} : An introduction to the basics of requirements analysis, including use cases, domain modeling, and UML.
			\item \textbf{\autoref{chapter:architectural-design} Architectural design} : Introduction on how to go from functional and non-functional requirements to a software architecture that meets these requirements.
			\item \textbf{\autoref{chapter:design-patterns} GRASP and design patterns} : Introduction to GRASP and an overview of some basic design patterns for software implementation.
			\item \textbf{\autoref{chapter:hci} Human-computer interaction} : Introduction to the field of human-computer interaction, usability and user interface design.
		\end{enumerate}
\end{enumerate}




\section*{Usage}

This text is aimed in particular at IT students so they would have some kind of summary or work of reference of most of the topics they will encounter throughout their study. Note that this document is not intended as a replacement for any of the course material used at universities or other instances of higher education.

For other readers, this document can be seen as a collection of independent topics or as a full-fledged course, teaching some of the basics on information technology. People that specialize in fields such as mathematics, physics, and economics among others, may find one or more chapters relating to those fields, that can act as a stepping stone to move on to the next level within that subject.

The list of referenced works may also be of interest to certain readers. Many of the books and articles referenced are used in academic courses.